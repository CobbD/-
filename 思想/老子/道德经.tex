% Created 2015-12-16 三 12:17
\documentclass[11pt,oneside]{article}
\usepackage{}
\usepackage{article}
\author{老子}
\date{\today}
\title{道德经}
\hypersetup{
 pdfauthor={老子},
 pdftitle={道德经},
 pdfkeywords={},
 pdfsubject={制作者邮箱:a358003542@gmail.com},
 pdfcreator={编者:wanze(<a href="mailto:a358003542@163.com">a358003542@163.com</a>)}, 
 pdflang={English}}
\begin{document}

\maketitle

\section{}
\label{sec:orgheadline1}
道可道,非常道。名可名,非常名。無名天地之始,有名萬物之母。故常無欲,以觀其妙;常有欲,以觀其徼。此兩者同出而異名,同謂之玄,玄之又玄,眾妙之門。

\section{}
\label{sec:orgheadline2}
天下皆知美之為美,斯惡已。皆知善之為善,斯不善已。故有無相生,難易相成,長短相較,高下相傾,音聲相和,前後相隨。是以聖人處無為之事,行不言之教;萬物作焉而不辭,生而不有,為而不恃,功成而弗居。夫唯弗居,是以不去。

\section{}
\label{sec:orgheadline3}
不尚賢,使民不爭;不貴難得之貨,使民不為盜;不見可欲,使民心不亂。是以聖人之治,虛其心,實其腹,弱其志,強其骨。常使民無知無欲。使夫智者不敢為也。為無為,則無不治。

\section{}
\label{sec:orgheadline4}
道沖而用之或不盈,淵兮似萬物之宗;挫其銳,解其紛,和其光,同其塵,湛兮似或存。吾不知誰之子,象帝之先。

\section{}
\label{sec:orgheadline5}
天地不仁,以萬物為芻狗;聖人不仁,以百姓為芻狗。天地之間,其猶橐籥乎﹖虛而不屈,動而愈出。多言數窮,不如守中。

\section{}
\label{sec:orgheadline6}
谷神不死,是謂玄牝。玄牝之門,是謂天地根。綿綿若存,用之不勤。

\section{}
\label{sec:orgheadline7}
天長地久。天地所以能長且久者,以其不自生,故能長生。是以聖人後其身而身先;外其身而身存。非以其無私邪,故能成其私。

\section{}
\label{sec:orgheadline8}
上善若水。水善利萬物而不爭,處眾人之所惡,故幾於道。居善地,心善淵,與善仁,言善信,正善治,事善能,動善時。夫唯不爭,故無尤。

\section{}
\label{sec:orgheadline9}
持而盈之,不如其已;揣而梲之,不可長保。金玉滿堂,莫之能守;富貴而驕,自遺其咎。功成身退,天之道也。

\section{}
\label{sec:orgheadline10}
載營魄抱一,能無離乎﹖專氣致柔,能嬰兒乎﹖滌除玄覽,能無疵乎﹖愛國治民,能無知乎﹖天門開闔,能為雌乎﹖明白四達,能無為乎﹖生之,畜之。生而不有,為而不恃,長而不宰,是謂玄德。

\section{}
\label{sec:orgheadline11}
三十輻,共一轂,當其無,有車之用。埏埴以為器,當其無,有器之用。鑿戶牖以為室,當其無,有室之用。故有之以為利,無之以為用。

\section{}
\label{sec:orgheadline12}
五色令人目盲,五音令人耳聾,五味令人口爽,馳騁畋獵令人心發狂,難得之貨令人行妨。是以聖人為腹不為目,故去彼取此。


\section{}
\label{sec:orgheadline13}
寵辱若驚,貴大患若身。何謂寵辱若驚﹖寵為下,得之若驚,失之若驚,是謂寵辱若驚。何謂貴大患若身﹖吾所以有大患者,為吾有身,及吾無身,吾有何患﹖故貴以身為天下,若可寄天下;愛以身為天下,若可托天下。

\section{}
\label{sec:orgheadline14}
視之不見名曰夷,聽之不聞名曰希,搏之不得名曰微。此三者不可致詰,故混而為一。其上不皦,其下不昧。繩繩不可名,復歸於無物。是謂無狀之狀,無物之象,是謂惚恍。迎之不見其首,隨之不見其後。執古之道,以御今之有。能知古始,是謂道紀。

\section{}
\label{sec:orgheadline15}
古之善為士者,微妙玄通,深不可識。夫唯不可識,故強為之容﹕豫兮若冬涉川,猶兮若畏四鄰,儼兮其若客,渙兮若冰之將釋,敦兮其若樸,曠兮其若谷,渾兮其若濁。孰能濁以靜之徐清﹖孰能安以久動之徐生﹖保此道者不欲盈,夫唯不盈,故能蔽不新成。

\section{}
\label{sec:orgheadline16}
致虛極,守靜篤。萬物並作,吾以觀復。夫物芸芸,各復歸其根。歸根曰靜,是曰復命。復命曰常,知常曰明。不知常,妄作凶。知常容,容乃公,公乃王,王乃天,天乃道,道乃久,沒身不殆。

\section{}
\label{sec:orgheadline17}
太上,下知有之,其次親而譽之,其次畏之,其次侮之。信不足焉,有不信焉。悠兮其貴言,功成事遂,百姓皆謂我自然。

\section{}
\label{sec:orgheadline18}
大道廢,有仁義;智慧出,有大偽;六親不和,有孝慈;國家昏亂,有忠臣。

\section{}
\label{sec:orgheadline19}
絕聖棄智,民利百倍;絕仁棄義,民復孝慈;絕巧棄利,盜賊無有。此三者以為文不足,故令有所屬﹕見素抱樸,少私寡欲。

\section{}
\label{sec:orgheadline20}
絕學無憂,唯之與阿,相去幾何﹖善之與惡,相去若何﹖人之所畏,不可不畏。荒兮其未央哉﹗眾人熙熙,如享太牢,如春登台。我獨泊兮,其未兆,如嬰兒之未孩;儡儡兮,若無所歸。眾人皆有餘,而我獨若遺。我愚人之心也哉﹗ 沌沌兮,俗人昭昭,我獨若昏。俗人察察,我獨悶悶。澹兮其若海,飂兮若無止。眾人皆有以,而我獨頑似鄙。我獨異於人,而貴食母。

\section{}
\label{sec:orgheadline21}
孔德之容,惟道是從。道之為物,惟恍惟惚。惚兮恍兮,其中有象;恍兮惚兮,其中有物。窈兮冥兮,其中有精;其精甚真,其中有信。自今及古,其名不去,以閱眾甫。吾何以知眾甫之狀哉﹖以此。

\section{}
\label{sec:orgheadline22}
曲則全,枉則直,窪則盈,敝則新,少則得,多則惑。是以聖人抱一為天下式。不自見,故明;不自是,故彰;不自伐,故有功;不自矜,故長。夫唯不爭,故天下莫能與之爭。古之所謂曲則全者,豈虛言哉﹗誠全而歸之。

\section{}
\label{sec:orgheadline23}
希言自然。故飄風不終朝,驟雨不終日。孰為此者﹖天地。天地尚不能久,而況於人乎﹖故從事於道者,(道者)同於道,德者同於德,失者同於失。同於道者,道亦樂得之;同於德者,德亦樂得之;同於失者,失亦樂得之。信不足焉,有不信焉。

\section{}
\label{sec:orgheadline24}
企者不立,跨者不行,自見者不明,自是者不彰,自伐者無功,自矜者不長。其在道也,曰「餘食贅行」。物或惡之,故有道者不處。

\section{}
\label{sec:orgheadline25}
有物混成,先天地生。寂兮寥兮,獨立而不改,周行而不殆,可以為天下母。吾不知其名,字之曰道,強為之名,曰大。大曰逝,逝曰遠,遠曰反。故道大,天大,地大,王亦大。域中有四大,而王居其一焉。人法地,地法天,天法道,道法自然。

\section{}
\label{sec:orgheadline26}
重為輕根,靜為躁君。是以聖人終日行不離輜重。雖有榮觀,燕處超然。奈何萬乘之主,而以身輕天下﹖輕則失本,躁則失君。

\section{}
\label{sec:orgheadline27}
善行無轍跡,善言無瑕謫;善數不用籌策;善閉無關楗而不可開,善結無繩約而不可解。是以聖人常善救人,故無棄人;常善救物,故無棄物,是謂襲明。故善人者,不善人之師;不善人者,善人之資。不貴其師,不愛其資,雖智大迷,是謂要妙。

\section{}
\label{sec:orgheadline28}
知其雄,守其雌,為天下谿。為天下谿,常德不離,復歸於嬰兒。知其白,守其黑,為天下式。為天下式,常德不忒,復歸於無極。知其榮,守其辱,為天下谷,常德乃足,復歸於樸。樸散則為器,聖人用之,則為官長,故大制不割。

\section{}
\label{sec:orgheadline29}
將欲取天下而為之,吾見其不得已。天下神器,不可為也,(不可執也。)為者敗之,執者失之。(是以聖人無為,故無敗;無執,故無失。)故物或行或隨;或歔或吹;或強或羸;或挫或隳。是以聖人去甚,去奢,去泰。

\section{}
\label{sec:orgheadline30}
以道佐人主者,不以兵強天下。其事好遠。師之所處,荊棘生焉。大軍之後,必有凶年。善者果而已,不以取強。果而勿矜,果而勿伐,果而勿驕。果而不得已,果而勿強。物壯則老,是謂不道,不道早已。


\section{}
\label{sec:orgheadline31}
夫〔佳〕兵者,不祥之器,物或惡之,故有道者不處。君子居則貴左,用兵則貴右。兵者不祥之器,非君子之器,不得已而用之,恬淡為上。勝而不美,而美之者,是樂殺人。夫樂殺人者,則不可以得志於天下矣。吉事尚左,凶事尚右。偏將軍居左,上將軍居右,言以喪禮處之。殺人之眾,以哀悲泣之,戰勝,以喪禮處之。

\section{}
\label{sec:orgheadline32}
道常無名,樸雖小,天下莫能臣也。侯王若能守之,萬物將自賓。天地相合,以降甘露,民莫之令而自均。始制有名,名亦既有,夫亦將知止,知止所以不殆。譬道之在天下,猶川谷之於江海。

*
知人者智,自知者明。勝人者有力,自勝者強。知足者富。強行者有志。不失其所者久。死而不亡者壽。

\section{}
\label{sec:orgheadline33}
大道氾兮,其可左右。萬物恃之而生而不辭,功成不名有。衣養萬物而不為主,常無欲,可名於小;萬物歸焉而不為主,可名為大。以其終不自為大,故能成其大。

\section{}
\label{sec:orgheadline34}
執大象,天下往。往而不害,安平太。樂與餌,過客止。道之出口,淡乎其無味,視之不足見,聽之不足聞,用之不可既。

\section{}
\label{sec:orgheadline35}
將欲歙之,必固張之;將欲弱之,必固強之;將欲廢之,必固興之;將欲奪之,必固與之。是謂微明。柔弱勝剛強。魚不可脫於淵,國之利器不可以示人。

\section{}
\label{sec:orgheadline36}
道常無為而無不為。侯王若能守之,萬物將自化。化而欲作,吾將鎮之以無名之樸。無名之樸,夫亦將無欲。不欲以靜,天下將自定。

\section{}
\label{sec:orgheadline37}
上德不德,是以有德;下德不失德,是以無德。上德無為而無以為;下德為之而有以為。上仁為之而無以為;上義為之而有以為。上禮為之而莫之應,則攘臂而扔之。故失道而後德,失德而後仁,失仁而後義,失義而後禮。夫禮者,忠信之薄,而亂之首。前識者,道之華,而愚之始。是以大丈夫處其厚,不居其薄;處其實,不居其華。故去彼取此。

\section{}
\label{sec:orgheadline38}
昔之得一者,天得一以清,地得一以寧,神得一以靈,谷得一以盈,萬物得一以生,侯王得一以為天下貞。其致之,天無以清將恐裂,地無以寧將恐發,神無以靈將恐歇,谷無以盈將恐竭,萬物無以生將恐滅,侯王無以貴高將恐蹶。故貴以賤為本,高以下為基。是以侯王自稱孤、寡、不穀。此非以賤為本邪﹖非乎﹖故致數輿無輿。不欲琭琭如玉,珞珞如石。

\section{}
\label{sec:orgheadline39}
反者道之動,弱者道之用。天下萬物生於有,有生於無。

\section{}
\label{sec:orgheadline40}
上士聞道,勤而行之;中士聞道,若存若亡;下士聞道,大笑之。不笑,不足以為道。故建言有之﹕明道若昧,進道若退,夷道若纇,上德若谷,大白若辱,廣德若不足,建德若偷,質真若渝,大方無隅,大器晚成,大音希聲,大象無形,道隱無名。夫唯道,善貸且成。

\section{}
\label{sec:orgheadline41}
道生一,一生二,二生三,三生萬物。萬物負陰而抱陽,沖氣以為和。人之所惡,唯孤、寡、不穀,而王公以為稱。故物或損之而益,或益之而損。人之所教,我亦教之。強梁者不得其死,吾將以為教父。

\section{}
\label{sec:orgheadline42}
天下之至柔,馳騁天下之至堅。無有入無閒,吾是以知無為之有益。不言之教,無為之益,天下希及之。

\section{}
\label{sec:orgheadline43}
名與身孰親﹖身與貨孰多﹖得與亡孰病﹖是故甚愛必大費,多藏必厚亡,知足不辱,知止不殆,可以長久。

\section{}
\label{sec:orgheadline44}
大成若缺,其用不弊。大盈若沖,其用不窮。大直若屈,大巧若拙,大辯若訥。靜勝躁,寒勝熱。清靜為天下正。

\section{}
\label{sec:orgheadline45}
天下有道,卻走馬以糞。天下無道,戎馬生於郊。禍莫大於不知足;咎莫大於欲得。故知足之足,常足矣。

\section{}
\label{sec:orgheadline46}
不出戶,知天下;不窺牖,見天道。其出彌遠,其知彌少。是以聖人不行而知,不見而明,不為而成。

\section{}
\label{sec:orgheadline47}
為學日益,為道日損。損之又損,以至於無為。無為而無不為。取天下常以無事,及其有事,不足以取天下。

\section{}
\label{sec:orgheadline48}
聖人無常心,以百姓心為心。善者,吾善之;不善者,吾亦善之;德善。信者,吾信之;不信者,吾亦信之;德信。聖人在,天下歙歙焉,為天下渾其心,百姓皆注其耳目,聖人皆孩之。

\section{}
\label{sec:orgheadline49}
出生入死。生之徒,十有三;死之徒,十有三;人之生,動之死地,亦十有三。夫何故﹖以其生生之厚。蓋聞善攝生者,陸行不遇兕虎,入軍不被甲兵;兕無所投其角,虎無所措其爪,兵無所容其刃。夫何故﹖以其無死地。

\section{}
\label{sec:orgheadline50}
道生之,德畜之,物形之,勢成之。是以萬物莫不尊道而貴德。道之尊,德之貴,夫莫之命而常自然。故道生之,德畜之。長之育之,亭之毒之,養之覆之。生而不有,為而不恃,長而不宰。是謂玄德。

\section{}
\label{sec:orgheadline51}
天下有始,以為天下母。既得其母,以知其子,既知其子,復守其母,沒身不殆。塞其兌,閉其門,終身不勤。開其兌,濟其事,終身不救。見小曰明,守柔曰強。用其光,復歸其明,無遺身殃,是為習常。

\section{}
\label{sec:orgheadline52}
使我介然有知,行於大道,唯施是畏。大道甚夷,而人好徑。朝甚除,田甚蕪,倉甚虛;服文綵,帶利劍,厭飲食,財貨有餘;是為夸盜。非道也哉﹗

\section{}
\label{sec:orgheadline53}
善建者不拔,善抱者不脫,子孫以祭祀不輟。修之於身,其德乃真;修之於家,其德乃餘;修之於鄉,其德乃長;修之於國,其德乃豐;修之於天下,其德乃普。故以身觀身,以家觀家,以鄉觀鄉,以國觀國,以天下觀天下。吾何以知天下然哉﹖以此。

\section{}
\label{sec:orgheadline54}
含德之厚,比於赤子。蜂蠆虺蛇不螫,猛獸不據,攫鳥不搏。骨弱筋柔而握固。未知牝牡之合而全作,精之至也。終日號而不嗄,和之至也。知和曰常,知常曰明。益生曰祥。心使氣曰強。物壯則老,謂之不道,不道早已。

\section{}
\label{sec:orgheadline55}
知者不言,言者不知。塞其兌,閉其門,挫其銳,解其分,和其光,同其塵,是謂玄同。故不可得而親,不可得而疏;不可得而利,不可得而害;不可得而貴,不可得而賤。故為天下貴。

\section{}
\label{sec:orgheadline56}
以正治國,以奇用兵,以無事取天下。吾何以知其然哉﹖以此。天下多忌諱,而民彌貧;民多利器,國家滋昏;人多伎巧,奇物滋起;法令滋彰,盜賊多有。故聖人云﹕「我無為,而民自化;我好靜,而民自正;我無事,而民自富;我無欲,而民自樸。」

\section{}
\label{sec:orgheadline57}
其政悶悶,其民淳淳;其政察察,其民缺缺。禍兮福之所倚,福兮禍之所伏。孰知其極﹖其無正。正復為奇,善復為妖。人之迷,其日固久。是以聖人方而不割,廉而不劌,直而不肆,光而不燿。

\section{}
\label{sec:orgheadline58}
治人事天,莫若嗇。夫唯嗇,是謂早服;早服謂之重積德;重積德則無不克;無不克則莫知其極;莫知其極,可以有國;有國之母,可以長久;是謂深根固柢,長生久視之道。

\section{}
\label{sec:orgheadline59}
治大國,若烹小鮮。以道蒞天下,其鬼不神;非其鬼不神,其神不傷人;非其神不傷人,聖人亦不傷人。夫兩不相傷,故德交歸焉。

\section{}
\label{sec:orgheadline60}
大國者下流,天下之交。天下之牝,牝常以靜勝牡,以靜為下。故大國以下小國,則取小國;小國以下大國,則取大國。故或下以取,或下而取。大國不過欲兼畜人,小國不過欲入事人。夫兩者各得其所欲,大者宜為下。

\section{}
\label{sec:orgheadline61}
道者萬物之奧。善人之寶,不善人之所保。美言可以市,尊行可以加人。人之不善,何棄之有﹖故立天子,置三公,雖有拱璧以先駟馬,不如坐進此道。 古之所以貴此道者何﹖不曰﹕以求得,有罪以免邪﹖故為天下貴。

\section{}
\label{sec:orgheadline62}
為無為,事無事,味無味。大小多少,報怨以德。圖難於其易,為大於其細;天下難事必作於易,天下大事必作於細。是以聖人終不為大,故能成其大。夫輕諾必寡信,多易必多難。是以聖人猶難之,故終無難矣。

\section{}
\label{sec:orgheadline63}
其安易持,其未兆易謀。其脆易泮,其微易散。為之於未有,治之於未亂。合抱之木,生於毫末;九層之臺,起於累土;千里之行,始於足下。為者敗之,執者失之。是以聖人無為故無敗,無執故無失。民之從事,常於幾成而敗之。慎終如始,則無敗事。是以聖人欲不欲,不貴難得之貨;學不學,復眾人之所過。以輔萬物之自然,而不敢為。

\section{}
\label{sec:orgheadline64}
古之善為道者,非以明民,將以愚之。民之難治,以其智多。故以智治國,國之賊;不以智治國,國之福。知此兩者亦稽式。常知稽式,是謂玄德。玄德深矣,遠矣,與物反矣,然後乃至大順。

\section{}
\label{sec:orgheadline65}
江海所以能為百谷王者,以其善下之,故能為百谷王。是以欲上民,必以言下之。欲先民,必以身後之。是以聖人處上而民不重,處前而民不害。是以天下樂推而不厭,以其不爭,故天下莫能與之爭。

\section{}
\label{sec:orgheadline66}
天下皆謂我道大,似不肖。夫唯大,故似不肖。若肖,久矣其細也夫﹗我有三寶,持而保之。一曰慈,二曰儉,三曰不敢為天下先。慈故能勇;儉故能廣;不敢為天下先,故能成器長。今舍慈且勇,舍儉且廣,舍後且先,死矣﹗夫慈以戰則勝,以守則固。天將救之,以慈衛之。

\section{}
\label{sec:orgheadline67}
善為士者不武,善戰者不怒,善勝敵者不與,善用人者為之下,是謂不爭之德,是謂用人之力,是謂配天古之極。

\section{}
\label{sec:orgheadline68}
用兵有言﹕「吾不敢為主而為客,不敢進寸而退尺。」是謂行無行,攘無臂,扔無敵,執無兵。禍莫大於輕敵,輕敵幾喪吾寶。故抗兵相加,哀者勝矣。

\section{}
\label{sec:orgheadline69}
吾言甚易知,甚易行。天下莫能知,莫能行。言有宗,事有君。夫唯無知,是以不我知。知我者希,則我者貴。是以聖人被褐懷玉。

\section{}
\label{sec:orgheadline70}
知不知上,不知知病。夫唯病病,是以不病。聖人不病,以其病病,是以不病。

\section{}
\label{sec:orgheadline71}
民不畏威,則大威至。無狎其所居,無厭其所生。夫唯不厭,是以不厭。是以聖人自知不自見;自愛不自貴。故去彼取此。

\section{}
\label{sec:orgheadline72}
勇於敢則殺,勇於不敢則活。此兩者,或利或害。天之所惡,孰知其故?是以聖人猶難之。天之道,不爭而善勝,不言而善應,不召而自來,繟然而善謀。天網恢恢,疏而不失。

\section{}
\label{sec:orgheadline73}
民不畏死,奈何以死懼之?若使民常畏死,而為奇者,吾得執而殺之,孰敢?常有司殺者殺。夫代司殺者殺,是謂代大匠斲,夫代大匠斲者,希有不傷其手矣。

\section{}
\label{sec:orgheadline74}
民之饑,以其上食稅之多,是以饑。民之難治,以其上之有為,是以難治。民之輕死,以其求生之厚,是以輕死。夫唯無以生為者,是賢於貴生。

\section{}
\label{sec:orgheadline75}
人之生也柔弱,其死也堅強。萬物草木之生也柔脆,其死也枯槁。故堅強者死之徒,柔弱者生之徒。是以兵強則不勝,木強則兵。強大處下,柔弱處上。

\section{}
\label{sec:orgheadline76}
天之道,其猶張弓與﹖高者抑之,下者舉之;有餘者損之,不足者補之。天之道,損有餘而補不足。人之道則不然,損不足以奉有餘。孰能有餘以奉天下,唯有道者。是以聖人為而不恃,功成而不處,其不欲見賢。

\section{}
\label{sec:orgheadline77}
天下莫柔弱於水,而攻堅強者莫之能勝,以其無以易之。弱之勝強,柔之勝剛,天下莫不知莫能行。是以聖人云﹕「受國之垢,是謂社稷主;受國不祥,是為天下王。」正言若反。

\section{}
\label{sec:orgheadline78}
和大怨,必有餘怨,安可以為善﹖是以聖人執左契,而不責於人。有德司契,無德司徹。天道無親,常與善人。

\section{}
\label{sec:orgheadline79}
小國寡民。使有什伯之器而不用,使民重死而不遠徙。雖有舟輿,無所乘之,雖有甲兵,無所陳之。使人復結繩而用之,甘其食,美其服,安其居,樂其俗。鄰國相望,雞犬之聲相聞,民至老死,不相往來。

\section{}
\label{sec:orgheadline80}
信言不美,美言不信。善者不辯,辯者不善。知者不博,博者不知。聖人不積,既以為人己愈有,既以與人己愈多。天之道,利而不害;聖人之道,為而不爭。
\end{document}
