% Created 2015-02-01 日 13:39
\documentclass[11pt,oneside]{article}
\usepackage{}
\usepackage{article}
\author{万泽}
\date{\today}
\title{亚当二号}
\hypersetup{
  pdfkeywords={},
  pdfsubject={制作者邮箱:a358003542@gmail.com},
  pdfcreator={编者:万泽}}
\begin{document}

\maketitle
\tableofcontents



\section{亚当二号}
\label{sec-1}
灵感来自文明4中的克隆工人

人们回溯这一切是怎么开始的,要追踪到二十世纪生物科学的蓬勃发展,DNA双螺旋结构的发现宣告了这样一个事实,人没有更多的秘密了,这就是人的全部了。

不过事情的发展证明上帝的智慧是无人能及的,在长时间的破解密码工作之后,人们开始又兴奋而变得冷静起来,一时之间,在那复杂的天书面前,人们发现要读懂它竟然才是一个真正的问题。

而本故事的开端则是从一个黑帮说起,说到黑帮,人们自然想到持枪抢劫贩毒什么的。而现在的我们要说的黑帮与其说是黑帮更像是一个公司,只是这个公司搞搞垄断啊搞搞关系啊什么的,当然了他们经营的业务有的时候也是见不得光的。不是有新闻说南非的黑帮控制着那里的一种稀有金属的生产,全世界的IT行业都托他们的福,还有什么美国的黑帮都参与那里的医保行业,因为这个行业的高利润。总之黑帮的黑是别人说的,要赚钱才是正经事。

什么市场利润高呢,供求严重失衡的市场。器官移植这个市场就是这个市场,器官移植这项手术,只是缝缝补补成本并不高,成本高的真正原因是要找到一个合适的器官实在是太困难了。我们上面说的那个黑帮就是负责转运人体器官的。当然了这个转运有的时候也包括从活体-用他们的术语-上取得。

不过这个黑帮最近生意好做得不得了了。因为他们联系了一家地下科研机构,这家科研机构也不知道是哪个疯狂博士开的。按照他的说法是本来人体器官的生产就不是什么坏事,而是一件天大的好事,一来可以救人性命;二来还可以防止不必要的活体事件。

这家黑帮给了那家科研机构资金支持,而那家科研机构也是非常的争气,竟然取得了某项世界上的领先技术,总之就是人体器官培育是可行的了。黑帮-科研机构-医院在地下发动了一场革命。而活在太阳底下的人们竟对此浑然不知,至于政府的限制令,现在竟然成为对他们有利的东西了,恰恰因为政府的限制,让这个行业少有竞争者,所以这个行业的利润变得一下子非常炙手可热起来。马上什么黑手党啊,斧头帮啊什么的都来分一杯羹,以至于那个时候有了这么一句话做黑帮的要是连器官移植都没有听说过就OUT了。

虽然当初只是打算做器官移植,但是明眼人都明白,下一步就是培育有钱人的克隆全体了。首先克隆全体培育保存都十分的方便,其次克隆全体在想用那部分就用那部分十分的方便,而单个的器官移植有的时候器官还没培育好人就死了,这个钱就没有赚到了,非常的可惜。所以那家科研机构正式启动了亚当二号的培育计划。

亚当二号的培育非常顺利,顺利的有点让人不知所措。哪位博士甚至觉得有点无聊起来,总之就是让克隆细胞继续深度分裂即可。不过后面的工作才开始遇到了问题,似乎是由于缺少母体环境的某种东西,早期的亚当二号不能进行肺部呼吸,不能进食,必须泡在营养液里,机器配合进行血液循环,甚至连心跳这样起码的生理机能都没有。这里躺着的就好像仅仅只是一大群进行了某种分化的细胞集团,他们必须之间完全没有交流和配合。那东西不过是个死东西罢了。

这种现状让博士觉得非常的不满意,他心里想至少这东西能够自己进行心跳,哪怕是个植物人维护成本也会大大的降低啊。

在进行了多次实验之后博士知道问题出在哪里了,问题出在大脑里面,似乎亚当二号的大脑没有任何活动,虽然那里的细胞也在进行着生化反应。事情接下来的进展多少有点出乎大家的意料之外,博士电击了亚当二号的大脑后部,亚当二号竟然开始有了心跳。博士兴奋不已。又多电击了几次,马上换上新的装备。

羊水被放走了,奶嘴一样的东西伸了下来。博士满心希望接下来那个东西会哭会呼吸会自己吸奶嘴。可是什么?什么也没有发生。那个东西不光没有呼吸,更不要说吸奶嘴了,连之前闪烁一下的心跳也没有了。怎么回事?

“亚当二号死了吗?”博士自言自语道。

正在绝望的时候,突然博士意识到一个重要的事实。电击大脑后部有了心跳,没有电击就没了。如果我建造一个设备持续电击呢。

“快!把亚当二号保存好。”

还是和之前一样的营养液,还是那个成人大小却什么都不会的一堆死肉。不同上次的是亚当二号的头部被植入了一个片状的可提供电击服务的电路板。这是最新的科技,那么一块微小的电路板的运算速度是以前通用的P4电脑的上万倍。而且还能通外界接受信号等等。

博士信心满满,一一问候工作人员之后,他望了望那个脑袋后面似乎打了一个补丁样东西的新的亚当二号。

“进入心跳电击测试阶段”
\ldots{}\ldots{}

常年累月的工作,博士获得的比想象中的还要多。心跳,呼吸,甚至包括流汗起鸡皮疙瘩这样的生理反应原来都可以追溯到大脑某部位的特殊信号的电击。常常的数据库更新了又更新。还有macrosoft公司一次又一次的补丁升级程序。使得亚当二号现在躺在那里能够像一个活着的人一样的躺在那里。

但是博士知道,那只不过是一个生化机器人罢了。他以前是看不起那些造机器人的,认为钢铁从来是不可能会思考的。而现在具有讽刺意味的是,他现在是一次又一次的借鉴那些造机器人的技术。他不得不接受这样一个事实,那的确只是一个简单的生化机器人罢了。我可以让亚当二号走路,可以让亚当二号吃饭,可以让亚当二号手指非常精确的运动。但是这所有的一切都只不过是亚当二号大脑后面的那个电路板里面的程序在其作用罢了。虽然和钢铁机器人具体实现的程序有所不同,但甚至这些程序的框架和结构基本原理都是一样的。博士知道自己造了一个什么。不是一个生命,只有上帝才可以创造生命,而我只不过造了一个生化机器人罢了。

虽然博士作为一个科学家的热情受到了凉水的冷静,但是他作为一个生意人的直觉马上让他兴奋起来,他知道他要发财了。

钢铁做的机器人造价非常昂贵,现在还只是某个富人的玩具或者某个超市吸引顾客的手段。生化机器人造价非常便宜,甚至比人还便宜。人从出生大大小小的开销还有上学的开销等等,而且要到18岁或者更久才能正式工作。而且他们还经常抱怨着抱怨那,偷懒,惹是生非。

“对!就叫他生化机器人,虽然可能刚开始人们不能接受。”

事实是明摆着的,生化机器人的竞争优势太明显了。第一份订单就是富士康公司。他足足给了这家科研机构二十个亿。他们的老板说他已经受够了那些人类了,他要全部将他的工人全部替换成生化机器人。

\ldots{}\ldots{}


2046年,中国富士康公司旗下所有员工发动了一场声势浩大的罢工活动,反抗这家公司对员工的盘扣剥夺。不过出乎大家意料之外的是,富士康辞退了全部所有的基层员工。罢工的工人们都有点摸不着头脑了,难道富士康不想干了吗?难道他不想赚钱了吗?

后来富士康所有的公司几乎实行军事化管理,或者说是更严格的军事化管理。人们只能从他的门口远远望着,没有人,是啊,没有人在里面干活。如果他们记得不错的话,可是他是只看见运送货物的车进进出出。而富士康赚的钱听说最近猛增。让股民们都为之疯狂。人们搞不懂,有好事的记者一次一次试图破解人们心中的困惑但都被赶了出去。有的甚至是惨遭殴打。

但是不久这样一个消息传了开来,说富士康里面有好多戴着头盔,身上披着编号的鬼魅在那里夜以继日的工作。外面传的沸沸扬扬,可是富士康也不出面澄清,只是说这是商业机密。你们无权过问。

又是一次严重的经济危机,人们似乎都只是这样平淡地述说着。相反世界末日即将来临的预言却在人群中激起了波澜,大家都异乎寻常地兴奋地谈论着这个以及这个该诅咒的社会。

欧洲区和美国的失业率再创新高,20\%,而中国则依然是4.7\%。但是民间的保守估算则是25\%-30\%了。还有一个现象更加引人堪忧,那就是自杀率也是不断创历史新高。一些更加离奇的事情出现了但人们似乎也是司空见惯了。有钱的人在外面包个二奶听说都过时了,用他们的话说就是至少也要有那个几个女人在床上服侍你那才叫人生的乐趣。社会动荡不安,时不时的有抢劫和枪杀事件发生。

而这时出现了一个团体,他们迅速发展壮大,他们说他们已经掌握了那些可耻的公司在用克隆人干活的证据。他们游行他们示威,要求政府给个解释出来。最后政府的官员出面说,确实现在有很多公司在用克隆人干活,不过那些克隆人都没有思想,基本的推理想象能力都没有。他们只是些生化机器人罢了。这些克隆人干的都是最脏最累的活,这项技术为的是让人类能够过上更加有品质的生活。

真的没有思想吗?还是只是傀儡。

当这一事件曝光之后,人们开始热议那些克隆人真的没有思想吗,还是他们有思想只是被傀儡了。他们抢走了我们的工作,我们这些无产阶级难道最终只能靠政府的救济这样没有尊严地活着吗?什么都没有,只成为富人的玩偶和嘲笑的对象。

社会矛盾已经发展到焦灼不可缓和的地步,当人们愤怒地撕下亚当二号的头皮之后,人们听到了一声惨叫。然后看到亚当二号从地面上站了起来。。。。。。
% 编者:万泽
\end{document}