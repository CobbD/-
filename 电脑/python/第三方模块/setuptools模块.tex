% Created 2015-02-19 四 04:15
\documentclass[11pt,oneside]{article}
\usepackage{}
\usepackage{article}
\author{万泽}
\date{\today}
\title{setuptools模块}
\hypersetup{
  pdfkeywords={},
  pdfsubject={制作者邮箱:a358003542@gmail.com},
  pdfcreator={编者:万泽}}
\begin{document}

\maketitle
\tableofcontents

虽然官方内置distutils模块也能实现类似的功能,不过现在人们更常用的是第三方模块setuptools,相当于distutils模块的加强版,初学者推荐就使用setuptools模块。

\section{安装}
\label{sec-1}
安装就是先安装pip3:
\begin{Verbatim}
sudo apt-get install python3-pip
\end{Verbatim}


然后通过pip3来安装setuptools:
\begin{Verbatim}
sudo pip3 install setuptools
\end{Verbatim}


\section{setuptools基础}
\label{sec-2}
安装官方文档最简单的“setup.py”文件如下所示:
\begin{tcbpython}
from setuptools import setup, find_packages
setup(
    name = "HelloWorld",
    version = "0.1",
    packages = find_packages(),
)
\end{tcbpython}

第一行是从setuptools模块中引入setup函数和find\_packages函数。

setup函数接受一系列的字典值,下面就setup函数的一些字典值的含义慢慢道来:

\begin{description}
\item[{name}] 本软件的名字
\item[{version}] 本软件的版本号
\item[{author}] 本软件的作者
\item[{author\_email}] 本软件作者的邮箱
\item[{maintainer}] 本软件的维护者
\item[{maintainer\_email}] 本软件维护者的邮箱
\item[{contact}] 本软件的联系人。可以不写,则是维护者的名字,如果没有则是作者的名字。
\item[{contact\_email}] 本软件的联系人的邮箱,可以不写,则是维护者的邮箱,如果没有则是作者的邮箱。
\item[{license}] 本软件的license
\item[{url}] 本软件项目主页地址
\item[{description}] 本软件的简要描述
\item[{long\_description}] 本软件的完整描述,一般如下定义一行函数,然后读取本地目录下面README.md文件。
\end{description}

\begin{tcbpython}
import os.path
def read(fname):
    return open(os.path.join(os.path.dirname(__file__), fname)).read()
\end{tcbpython}

然后long\_description如下设置:
\begin{tcbpython}
long_description=read('README.md'),
\end{tcbpython}

\begin{description}
\item[{platforms}] 本软件经过测试可运行的平台
\item[{classifiers}] 本软件的分类,请参考\href{https://pypi.python.org/pypi?\%3Aaction=list_classifiers}{这个网页} 给出一些值。是字符串的列表。
\item[{keywords}] 本软件在pypi上搜索的关键词,字符串的列表。
\item[{packages}] 你的软件依赖的模块。一般如下使用:
\end{description}

\begin{tcbpython}
packages = find_packages(),
\end{tcbpython}

则你文件夹下有 \verb~__init__.py~ 文件的都将视作python模块并加入进去。

除此之外你也可以直接手工输入你的模块名字,具体就是字符串的列表。
\begin{description}
\item[{entry\_point}] entry\_point属性视图解决在Linux \footnote{windows下的情况不清楚。} 下的脚本调用问题。如下所示:
\end{description}

\begin{tcbpython}
entry_points = {
'console_scripts' :[ 'zwc=zwc.zwc:main',],
}
\end{tcbpython}

其中zwc是你的shell调用的名字,然后zwc是你的模块,另外一个zwc是你的主模块的子模块,然后main是其中的main函数。这就是你的shell调用程序的接口了。类似的还有gui\_script可以控制你调用GUI图形的命令入口。
\begin{description}
\item[{install\_requires}] 接受字符串的列表值,将你依赖的可以通过pip安装的模块名放入进去,然后你的软件安装会自动检测并安装这些依赖模块。
\item[{package\_data}] 你的软件的模块额外附加的(除了py文件的)其他文件,具体设置类似这样 \verb~{"skeleton":['*.txt'],}~ 其中skeleton这里就是具体的你的软件的模块(对应的文件夹名),然后后面跟着的就是一系列的文件名列表,可以接受glob语法。注意这里只能包含你的模块文件夹也就是前面通过packages控制的文件夹下面的内容。
\item[{include\_package\_data}] 这个一般设置为True,似乎可以省略?
\end{description}

其他不常用的属性值列在下面:
\begin{description}
\item[{scripts}] 不推荐使用,推荐通过entry\_point来生成脚本。
\item[{py\_modules}] 不推荐使用,推荐使用packages来管理模块。
\item[{data\_files}] 前面的package\_data是只能在你的模块文件夹里面的其他数据文件等,然后可能还有一些数据文件你需要包含的,用data\_files来控制,具体后面跟着的参数格式如下面例子所示:
\end{description}
\begin{tcbpython}
data_files = [('icos',['icos/wise.ico'])],
#这是添加的icos文件夹下面的wise.ico文件
data_files = [('',['skeleton.tar.gz'])],
#这是添加的主目录下的skeleton.tar.gz文件
\end{tcbpython}

值得一提的是data\_files不能接受glob语法。

data\_files已经不推荐使用了,推荐用package\_data来管理,可以方便用pkg\_resources里面的方法来引用其中的资源文件。具体说明请看后面。


\section{pkg\_resources模块来管理读取资源文件}
\label{sec-3}
如下所示
\begin{Verbatim}
from pkg_resources import resource_filename
resource_stream('wise','icos/Folder-Documents.ico')
\end{Verbatim}

第一个参数是模块名字,第二个参数是模块中的文件的相对路径表达。

上面的例子是resource\_filename,返回的是引用的文件名。此外还有命令:resource\_string,参数和resource\_filename一样,除了它返回的是字节流。这个字节流可以赋值给某个变量从而直接使用,或者存储在某个文件里面。


\section{在pypi上注册你的软件}
\label{sec-4}
具体很简单,就是
\begin{tcbbash}
sudo python3 setup.py register
\end{tcbbash}

你需要在pypi官网上注册一个帐号,然后你的软件不一定能够注册成功,因为很多好名字都被别人取了。。

\section{一个项目模板}
\label{sec-5}
我在github上创建了一个小项目,这个小项目支持快速新建一个项目。

\href{https://github.com/a358003542/skeleton}{\url{https://github.com/a358003542/skeleton}} 
% 编者:万泽
\end{document}
