% Created 2015-02-11 三 11:00
\documentclass[11pt,oneside]{article}
\usepackage{}
\usepackage{article}
\author{万泽}
\date{\today}
\title{opencv模块}
\hypersetup{
  pdfkeywords={},
  pdfsubject={制作者邮箱:a358003542@gmail.com},
  pdfcreator={编者:万泽}}
\begin{document}

\maketitle
\tableofcontents



\section{安装}
\label{sec-1}
\subsection{预安装}
\label{sec-1-1}
官网教程也提到了需要预安装的一些软件包:

\begin{Verbatim}
[compiler] sudo apt-get install build-essential
[required] sudo apt-get install cmake git libgtk2.0-dev pkg-config libavcodec-dev 
    libavformat-dev libswscale-dev
[optional] sudo apt-get install python-dev python-numpy libtbb2 libtbb-dev 
    libjpeg-dev libpng-dev libtiff-dev libjasper-dev libdc1394-22-dev
\end{Verbatim}

不过官网说这只在Ubuntu 10.04 中测试过,你还可以参考\href{http://rodrigoberriel.com/2014/10/installing-opencv-3-0-0-on-ubuntu-14-04/}{这篇文章} ,其是在Ubuntu 14.04 下做的测试:
\begin{Verbatim}
sudo apt-get -y install libopencv-dev build-essential cmake git 
    libgtk2.0-dev pkg-config python-dev python-numpy libdc1394-22 
    libdc1394-22-dev libjpeg-dev libpng12-dev libtiff4-dev libjasper-dev
    libavcodec-dev libavformat-dev libswscale-dev libxine-dev
    libgstreamer0.10-dev libgstreamer-plugins-base0.10-dev libv4l-dev 
    libtbb-dev libqt4-dev libfaac-dev libmp3lame-dev libopencore-amrnb-dev 
    libopencore-amrwb-dev libtheora-dev libvorbis-dev libxvidcore-dev x264 
    v4l-utils unzip
\end{Verbatim}
可以作为参考。

\subsection{下载源码和安装}
\label{sec-1-2}
opencv的github源码地址在 \href{https://github.com/Itseez/opencv}{这里} ,推荐下载稍微前面点的版本会稳定,因为opencv还在活跃开发中。

在切换到opencv文件之后,新建了一个build文件夹,然后调用cmake命令,我们可以看到cmake最后对应的文件夹 “..“ 正是要去找寻那个文件:CMakeLists.txt。
\begin{Verbatim}
cd opencv
mkdir build
cd build
cmake -D CMAKE_BUILD_TYPE=RELEASE -D CMAKE_INSTALL_PREFIX=/usr/local 
    -D BUILD_NEW_PYTHON_SUPPORT=ON  -D WITH_IPP=OFF  -D WITH_OPENGL=ON ..
\end{Verbatim}

不过这里cmake参数挺复杂的,其中
\begin{Verbatim}
-D CMAKE_BUILD_TYPE=RELEASE -D CMAKE_INSTALL_PREFIX=/usr/local
\end{Verbatim}
都会这么写,然后
\begin{Verbatim}
-D BUILD_NEW_PYTHON_SUPPORT=ON
\end{Verbatim}
很重要,也就是安装了对应的python模块。

然后我在安装的时候出现了一个错误,参考 \href{http://answers.opencv.org/question/37115/opencv-249-make-error/}{这个网页} 加上如下参数即可解决:
\begin{Verbatim}
-D WITH_IPP=OFF
\end{Verbatim}

然后rodrigoberriel\footnote{\href{http://rodrigoberriel.com/2014/10/installing-opencv-3-0-0-on-ubuntu-14-04/}{rodrigoberriel.com}} 提及如果如果要使用Qt5,则不能加上 \verb~-D WITH_QT=ON~ ,请参看 \href{http://rodrigoberriel.com/2014/11/using-opencv-3-qt-creator-3-2-qt-5-3/}{这篇文章} 。其他还有一些参数选项我暂时也不清楚有什么用,应该都和其他额外的功能有关,暂时都可加可不加吧。

然后常规的运行make 和 make install 来进行安装了,下面的 \verb~$(nproc)~ 对应的你机器的cpu数目。
\begin{Verbatim}
make -j $(nproc)
sudo make install
\end{Verbatim}

和官网不同,rodrigoberriel还提出需要加上如下命令来完成安装:
\begin{Verbatim}
sudo /bin/bash -c 'echo "/usr/local/lib" > /etc/ld.so.conf.d/opencv.conf'
sudo ldconfig
\end{Verbatim}


安装完了你可以简单来查看一下:
\begin{Verbatim}
>>> import cv2
>>> cv2.__version__
'3.0.0-beta'
\end{Verbatim}
% 编者:万泽
\end{document}
